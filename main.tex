\documentclass{article}
\usepackage[utf8]{inputenc}
\usepackage{amsmath}
\usepackage{amssymb}
\usepackage{graphicx}
\graphicspath{ {./images/} }

\title{LaTeX Assignment 0}
\author{Gibson Green }
\date{January 13, 2019}

\begin{document}

\maketitle

1. A 3 x 5 box with a title row

\vspace{5mm}
\begin{tabular}{|c|c|c|c|c|} 
 \hline
 Column 1 & Column 2 & Column 3 & Column 4 & Column 5 \\
 \hline
& & & & & & & & &\\ \hline
& & & & & & & & &\\ \hline
& & & & & & & & &\\ \hline
 \end{tabular}
 \vspace{5mm}
 
 2. Find the radius of convergence and interval of convergence of the series.
 
 $$\sum_{n=1}^{\infty} \frac{x^{n}}{n^{2}5^{n}}$$
 \textbf{Answers}
 \vspace{5mm}
 $$\lim_{n\to\infty} \mid \frac{a_{n+1}}{a_n} \mid = \lim_{n\to\infty} \mid \frac{x^{n+1}}{(n+1)^2 5^{n+1}} \cdot \frac{n^2 5^n}{x^n} \mid $$
 $$= \lim_{n\to\infty} \frac{1}{(1+\frac{1}{n})^2} \frac{\mid x \mid}{5}$$
 $$= \frac{\mid x \mid}{5} $$
 
\vspace{5mm}
By the Ratio Test, this series converges when $\frac{\mid x \mid}{5}<1,$ or $\mid x \mid <5$. Hence, the radius of convergence is 5. Checking the endpoints: When x= -5, the series is $\sum_{n=1}^{\infty} \frac{1}{n^2}$, which is a convergent p-series. When x=5, the series becomes $\sum_{n=1}^{\infty} \frac{(-1)^n}{n^2}$, which converges by the alternation series test. \\

\vspace{5mm}
Find the area of the region that lies inside both circle $r= 2sin(\theta) + cos(\theta).$ Hint: consider two regions. \\
 \textbf{Answers} \\
 The curves intersect where $2 sin(\theta)= sin(\theta)+cos(\theta) \Rightarrow{}sin(\theta)= cos(\theta) \Rightarrow{}\theta = \frac{\pi}{4}$, and also at the origin at which $\theta= \frac{3\pi}{4}$ on the second curve. \\

\vspace{5mm}
$$A= \int_{0}^{\frac{\pi}{4}} \frac{1}{2} (2sin\theta)^2 d\theta + \int_{\frac{\pi}{4}}^{\frac{3\pi}{4}} \frac{1}{2} (sin\theta + cos\theta)^2 d\theta $$
$$ = \int_{0}^{\frac{\pi}{4}} (1-cos2\theta) d\theta + \frac{1}{2} \int_{\frac{\pi}{4}}^{\frac{3\pi}{4}} (1 + sin2\theta) d\theta $$
$$= \Big[\theta - \frac{1}{2}sin2\theta \Big]^{\frac{\pi}{4}}_0 + 
\Big[\frac{1}{2}\theta - \frac{1}{4}cos2\theta \Big]^{\frac{3\pi}{4}}_\frac{\pi}{4}$$
$$= \frac{1}{2} (\pi - 1) $$

4. Create a 4 x 9 matrix.
\vspace{5mm}
\[
\begin{bmatrix}
    1 & 0 & 0 & 0 & 1 & 0 & 0 & 0 & 1 \\
    0 & 1 & 0 & 0 & 0 & 1 & 1 & 1 & 0  \\
    0 & 0 & 1 & 0 & 1 & 0 & 0 & 1 & 1 \\
    0 & 0 & 0 & 1 & 1 & 0 & 0 & 0 & 1 \\
\end{bmatrix}
\]
\vspace{5mm}
    \item 
5. Create an itemized list with an itemized sublist 
\begin{itemize}    
  \item (Submitted by Dewey) x=3 
  \item (Submitted by Cheatem) x = $\frac{{3}}{{4}}$ or x = $\cfrac{{3}}{{4}}$ \\
  \item (Submitted by Andy Howe)
  \begin{itemize}
    \item A regular, inline fraction x = $\frac{{3}}{{\frac{a}{0}}}$
    \item A fraction centered on a new line with larger size:
    $$ x = \cfrac{{3}}{{\frac{a}{0}}}$$
    \item A fraction, inline, but with larger size: x = $\cfrac{{3}}{{\cfrac{a}{0}}}$
     \end{itemize}
\end{itemize}

6. I can type \textbf{bold text}, {\textit{italicized text}}, the real 
or \newcommand{\R}{\mathbb{R}} \( \R \) \ using "backslash R" (since it is a "newcommand") and the integers \newcommand{\Z}{\mathbb{Z}} \( \Z \) \
I can even type x \in \newcommand{\N}{\mathbb{N}} \( \N \) \ or A  \cap B \subseteq C

 7. 
 \vspace{5mm}
 \includegraphics[width=\linewidth]{Me.png}

\end{document}
